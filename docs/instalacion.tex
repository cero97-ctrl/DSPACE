\documentclass[a4paper,11pt]{article}
\usepackage[utf8]{inputenc}
\usepackage[spanish]{babel}
\usepackage{listings}
\usepackage{xcolor}
\usepackage[hidelinks]{hyperref}
\usepackage{geometry}

% Configuración de márgenes
\geometry{
 a4paper,
 total={170mm,257mm},
 left=20mm,
 top=20mm,
}

% Definición de colores para los bloques de código
\definecolor{codegreen}{rgb}{0,0.6,0}
\definecolor{codegray}{rgb}{0.5,0.5,0.5}
\definecolor{codepurple}{rgb}{0.58,0,0.82}
\definecolor{backcolour}{rgb}{0.95,0.95,0.92}

% Estilo para los listados de código
\lstdefinestyle{mystyle}{
    backgroundcolor=\color{backcolour},
    commentstyle=\color{codegreen},
    keywordstyle=\color{magenta},
    numberstyle=\tiny\color{codegray},
    stringstyle=\color{codepurple},
    basicstyle=\ttfamily\footnotesize,
    breakatwhitespace=false,
    breaklines=true,
    captionpos=b,
    keepspaces=true,
    numbers=left,
    numbersep=5pt,
    showspaces=false,
    showstringspaces=false,
    showtabs=false,
    tabsize=2,
    literate={á}{{\'a}}1 {é}{{\'e}}1 {í}{{\'i}}1 {ó}{{\'o}}1 {ú}{{\'u}}1 {ñ}{{\~n}}1 {Á}{{\'A}}1 {É}{{\'E}}1 {Í}{{\'I}}1 {Ó}{{\'O}}1 {Ú}{{\'U}}1 {Ñ}{{\~N}}1
}

\lstset{style=mystyle}

\title{Guía de Instalación de DSpace 7 con Docker}
\author{Prof. César Rodríguez}
\date{\today}

\begin{document}

\maketitle

\tableofcontents
\newpage

\section{Introducción}
Esta guía describe el procedimiento para desplegar DSpace 7 utilizando contenedores Docker. Este método simplifica drásticamente la gestión de dependencias (Java, Maven, PostgreSQL, Solr) ya que todo se ejecuta dentro de contenedores aislados y preconfigurados.

\section{Prerrequisitos}
\begin{itemize}
    \item \textbf{Docker Engine} (v19.03 o superior).
    \item \textbf{Docker Compose} (v1.25 o superior).
    \item \textbf{Git}.
    \item Al menos 4GB de RAM dedicados a Docker (recomendado 8GB).
\end{itemize}

\section{Instalación y Despliegue}

\subsection{1. Descarga del Código Fuente}
Clonamos el repositorio principal de DSpace. Este repositorio incluye los archivos \texttt{docker-compose.yml} necesarios para levantar todo el stack (Backend, Frontend, BD y Solr).
\begin{lstlisting}[language=bash]
mkdir /dspace-docker
cd /dspace-docker
git clone https://github.com/DSpace/DSpace.git -b dspace-7.6
cd DSpace
\end{lstlisting}

\subsection{2. Despliegue de Contenedores}
Utilizamos Docker Compose para descargar las imágenes oficiales y levantar los servicios.
\begin{lstlisting}[language=bash]
# Descargar las imágenes más recientes
docker compose pull

# Levantar los servicios en segundo plano (detached mode)
docker compose -p dspace up -d
\end{lstlisting}

Esto iniciará los siguientes servicios automáticamente:
\begin{itemize}
    \item \textbf{dspace-angular}: Frontend (Puerto 4000).
    \item \textbf{dspace-server}: Backend API (Puerto 8080).
    \item \textbf{dspace-db}: Base de datos PostgreSQL.
    \item \textbf{dspace-solr}: Motor de búsqueda Solr.
\end{itemize}

\subsection{3. Inicialización de la Base de Datos}
Una vez que los contenedores estén corriendo, es necesario ejecutar las migraciones iniciales para crear el esquema de la base de datos.
\begin{lstlisting}[language=bash]
# Ejecutar migración de base de datos dentro del contenedor
docker compose -p dspace exec dspace-server dspace database migrate
\end{lstlisting}

\subsection{4. Creación del Administrador}
Para acceder al sistema y gestionar contenidos, creamos una cuenta de administrador (e-person).
\begin{lstlisting}[language=bash]
docker compose -p dspace exec dspace-server dspace create-administrator
\end{lstlisting}
Siga las instrucciones en pantalla para ingresar email, nombre, apellido y contraseña.

\section{Verificación}
Acceda a las siguientes URLs desde su navegador:
\begin{itemize}
    \item \textbf{Frontend (UI):} \url{http://localhost:4000}
    \item \textbf{Backend (API):} \url{http://localhost:8080/server}
    \item \textbf{HAL Browser:} \url{http://localhost:8080/server/api}
\end{itemize}

\section{Comandos Útiles}

\subsection{Ver logs}
\begin{lstlisting}[language=bash]
# Ver logs del backend en tiempo real
docker compose -p dspace logs -f dspace-server

# Ver logs del frontend
docker compose -p dspace logs -f dspace-angular
\end{lstlisting}

\subsection{Detener el sistema}
\begin{lstlisting}[language=bash]
docker compose -p dspace down
\end{lstlisting}

\section{Resumen de la Instalación}
Al completar estos pasos, usted tendrá efectivamente \textbf{DSpace 7.6} ejecutándose en su PC GNU/Linux bajo la siguiente arquitectura:
\begin{itemize}
    \item \textbf{Contenerización Total:} Backend, Frontend, Base de Datos y Solr corren aislados en Docker.
    \item \textbf{Persistencia:} Los datos se guardan en volúmenes de Docker, no se pierden al apagar el contenedor.
    \item \textbf{Versión:} Se garantiza la versión 7.6 al haber seleccionado la rama específica en el paso de clonación.
\end{itemize}

\end{document}