\documentclass[a4paper,11pt]{article}
\usepackage[utf8]{inputenc}
\usepackage[spanish]{babel}
\usepackage{listings}
\usepackage{xcolor}
\usepackage[hidelinks]{hyperref}
\usepackage{geometry}

% Configuración de márgenes
\geometry{
 a4paper,
 total={170mm,257mm},
 left=20mm,
 top=20mm,
}

% Definición de colores para los bloques de código
\definecolor{codegreen}{rgb}{0,0.6,0}
\definecolor{codegray}{rgb}{0.5,0.5,0.5}
\definecolor{codepurple}{rgb}{0.58,0,0.82}
\definecolor{backcolour}{rgb}{0.95,0.95,0.92}

% Estilo para los listados de código
\lstdefinestyle{mystyle}{
    backgroundcolor=\color{backcolour},
    commentstyle=\color{codegreen},
    keywordstyle=\color{magenta},
    numberstyle=\tiny\color{codegray},
    stringstyle=\color{codepurple},
    basicstyle=\ttfamily\footnotesize,
    breakatwhitespace=false,
    breaklines=true,
    captionpos=b,
    keepspaces=true,
    numbers=left,
    numbersep=5pt,
    showspaces=false,
    showstringspaces=false,
    showtabs=false,
    tabsize=2,
    literate={á}{{\'a}}1 {é}{{\'e}}1 {í}{{\'i}}1 {ó}{{\'o}}1 {ú}{{\'u}}1 {ñ}{{\~n}}1 {Á}{{\'A}}1 {É}{{\'E}}1 {Í}{{\'I}}1 {Ó}{{\'O}}1 {Ú}{{\'U}}1 {Ñ}{{\~N}}1
}

\lstset{style=mystyle}

\title{Manual Técnico y Guía de Instalación: DSpace 7+}
\author{Prof. César Rodríguez - UDO/SUCRE}
\date{\today}

\begin{document}

\maketitle

\tableofcontents
\newpage

\section{Introducción}
DSpace es el software de código abierto más utilizado a nivel mundial para la creación de repositorios institucionales y archivos digitales. Las versiones modernas (DSpace 7 y 8) representan un cambio arquitectónico significativo respecto a las versiones anteriores (4, 5, 6).

\section{Arquitectura Técnica (DSpace 7+)}
A partir de la versión 7, DSpace se dividió en una arquitectura \textbf{Headless} (Backend y Frontend separados):

\subsection{Backend (DSpace-Server)}
\begin{itemize}
    \item Basado en \textbf{Java} y \textbf{Spring Boot}.
    \item Expone una \textbf{REST API} completa (HAL browser compliant).
    \item Utiliza \textbf{Hibernate} para la persistencia de datos.
    \item Utiliza \textbf{Apache Solr} para la indexación y búsqueda.
    \item Base de datos: Generalmente \textbf{PostgreSQL} (o Oracle).
\end{itemize}

\subsection{Frontend (DSpace-UI)}
\begin{itemize}
    \item Basado en \textbf{Angular} (TypeScript).
    \item Utiliza \textbf{Node.js} para el Server-Side Rendering (SSR) para mejorar el SEO.
    \item Se comunica exclusivamente con el backend a través de la REST API.
\end{itemize}

\section{Estructura de Datos}
DSpace sigue una jerarquía estricta basada en el modelo Dublin Core:
\begin{enumerate}
    \item \textbf{Comunidades} (Communities)
    \item \textbf{Colecciones} (Collections)
    \item \textbf{Items} (Artículos/Tesis)
    \item \textbf{Bitstreams} (Archivos físicos como PDFs)
\end{enumerate}

\section{Guía de Instalación con Docker}
Esta sección describe el procedimiento para desplegar DSpace 7 utilizando contenedores Docker. Este método simplifica drásticamente la gestión de dependencias (Java, Maven, PostgreSQL, Solr) ya que todo se ejecuta dentro de contenedores aislados y preconfigurados.

\subsection{Prerrequisitos}
\begin{itemize}
    \item \textbf{Docker Engine} (v19.03 o superior).
    \item \textbf{Docker Compose} (v1.25 o superior).
    \item \textbf{Git}.
    \item Al menos 4GB de RAM dedicados a Docker (recomendado 8GB).
\end{itemize}

\subsection{Instalación y Despliegue}

\subsubsection{1. Descarga del Código Fuente}
Clonamos el repositorio principal de DSpace. Este repositorio incluye los archivos \texttt{docker-compose.yml} necesarios para levantar todo el stack (Backend, Frontend, BD y Solr).
\begin{lstlisting}[language=bash]
mkdir /dspace-docker
cd /dspace-docker
git clone https://github.com/DSpace/DSpace.git -b dspace-7.6
cd DSpace
\end{lstlisting}

\subsubsection{2. Despliegue de Contenedores}
Utilizamos Docker Compose para descargar las imágenes oficiales y levantar los servicios.
\begin{lstlisting}[language=bash]
# Descargar las imágenes más recientes
docker compose pull

# Levantar los servicios en segundo plano (detached mode)
docker compose -p dspace up -d
\end{lstlisting}

Esto iniciará los siguientes servicios automáticamente:
\begin{itemize}
    \item \textbf{dspace-angular}: Frontend (Puerto 4000).
    \item \textbf{dspace-server}: Backend API (Puerto 8080).
    \item \textbf{dspace-db}: Base de datos PostgreSQL.
    \item \textbf{dspace-solr}: Motor de búsqueda Solr.
\end{itemize}

\subsubsection{3. Inicialización de la Base de Datos}
Una vez que los contenedores estén corriendo, es necesario ejecutar las migraciones iniciales para crear el esquema de la base de datos.
\begin{lstlisting}[language=bash]
# Ejecutar migración de base de datos dentro del contenedor
docker compose -p dspace exec dspace-server dspace database migrate
\end{lstlisting}

\subsubsection{4. Creación del Administrador}
Para acceder al sistema y gestionar contenidos, creamos una cuenta de administrador (e-person).
\begin{lstlisting}[language=bash]
docker compose -p dspace exec dspace-server dspace create-administrator
\end{lstlisting}
Siga las instrucciones en pantalla para ingresar email, nombre, apellido y contraseña.

\subsection{Verificación}
Acceda a las siguientes URLs desde su navegador:
\begin{itemize}
    \item \textbf{Frontend (UI):} \url{http://localhost:4000}
    \item \textbf{Backend (API):} \url{http://localhost:8080/server}
    \item \textbf{HAL Browser:} \url{http://localhost:8080/server/api}
\end{itemize}

\subsection{Comandos Útiles}

\subsubsection{Ver logs}
\begin{lstlisting}[language=bash]
# Ver logs del backend en tiempo real
docker compose -p dspace logs -f dspace-server

# Ver logs del frontend
docker compose -p dspace logs -f dspace-angular
\end{lstlisting}

\subsubsection{Detener el sistema}
\begin{lstlisting}[language=bash]
docker compose -p dspace down
\end{lstlisting}

\subsection{Puesta en Producción (IP Pública)}
Cuando disponga de la IP pública asignada por Teleinformática, deberá actualizar la configuración para permitir el acceso externo.

\subsubsection{Ajustes Básicos}
\begin{itemize}
    \item \textbf{Backend:} Actualizar \texttt{dspace.server.url} y \texttt{dspace.ui.url} en \texttt{local.cfg}.
    \item \textbf{Frontend:} Configurar el cliente Angular para que apunte a la IP pública de la API (usualmente mediante variables de entorno \texttt{DSPACE\_REST\_HOST} en Docker).
\end{itemize}

\subsubsection{Proxy Reverso (Recomendado)}
Para un entorno de producción seguro, no exponga los puertos 8080/4000 directamente. Configure \textbf{Nginx} o \textbf{Apache} como proxy reverso en el servidor host para:
\begin{enumerate}
    \item Recibir tráfico en puertos estándar (80/443).
    \item Gestionar certificados SSL (HTTPS).
    \item Redirigir internamente a los contenedores Docker.
\end{enumerate}

\subsection{Resumen de la Instalación}
Al completar estos pasos, usted tendrá efectivamente \textbf{DSpace 7.6} ejecutándose en su PC GNU/Linux bajo la siguiente arquitectura:
\begin{itemize}
    \item \textbf{Contenerización Total:} Backend, Frontend, Base de Datos y Solr corren aislados en Docker.
    \item \textbf{Persistencia:} Los datos se guardan en volúmenes de Docker, no se pierden al apagar el contenedor.
    \item \textbf{Versión:} Se garantiza la versión 7.6 al haber seleccionado la rama específica en el paso de clonación.
\end{itemize}

\section{Ejemplos de Código y Configuración}

\subsection{Configuración Básica (local.cfg)}
En lugar de editar el \texttt{dspace.cfg} principal, las mejores prácticas de ingeniería dictan que debes sobreescribir las configuraciones en \texttt{local.cfg}.

\begin{lstlisting}[language=bash, caption=Configuración en local.cfg]
# dspace/config/local.cfg

# Configuración del directorio de instalación
dspace.dir = /dspace

# Configuración de la URL pública (Frontend Angular)
dspace.ui.url = https://repositorio.miuniversidad.edu.co

# Configuración de la REST API (Backend Java)
dspace.server.url = https://api.repositorio.miuniversidad.edu.co/server

# Conexión a Base de Datos (PostgreSQL)
db.url = jdbc:postgresql://localhost:5432/dspace
db.username = dspace
db.password = mi_password_seguro
db.schema = public

# Configuración de Solr (Búsqueda)
solr.server = http://localhost:8983/solr
\end{lstlisting}

\subsection{Desarrollo Backend (Java API)}
Si necesitas crear una tarea curada (Curation Task) o un script personalizado para manipular ítems programáticamente, interactuarás con los servicios del Kernel de DSpace.

\begin{lstlisting}[language=Java, caption=Procesamiento de Items en Java]
package org.dspace.example;

import java.sql.SQLException;
import java.util.UUID;
import org.dspace.content.Item;
import org.dspace.content.service.ItemService;
import org.dspace.core.Context;
import org.dspace.services.factory.DSpaceServicesFactory;

public class ItemProcessor {

    // Obtenemos el servicio de Items a través de la factoría de servicios
    protected ItemService itemService = DSpaceServicesFactory.getInstance().getItemService();

    public void processItem(Context context, String uuidString) {
        try {
            // Convertir String a UUID
            UUID uuid = UUID.fromString(uuidString);
            
            // Buscar el item usando el contexto actual
            Item item = itemService.find(context, uuid);

            if (item != null) {
                // Ejemplo: Leer metadatos (Dublin Core Title)
                String title = itemService.getMetadataFirstValue(
                    item, "dc", "title", null, Item.ANY
                );
                
                System.out.println("Procesando Item: " + title);
                
                // Aquí podrías añadir lógica para modificar metadatos o bitstreams
            }
        } catch (SQLException e) {
            e.printStackTrace();
        }
    }
}
\end{lstlisting}

\section{Implementación de Metadatos UDO (Análisis de PDFs)}
Basado en los documentos ``Hoja de Metadatos'' e ``Instructivo 2024'' de la Universidad de Oriente, se ha diseñado la siguiente estrategia de configuración para DSpace 7.6.

\subsection{1. Mapeo de Campos (Mapping)}
Se propone la siguiente correspondencia entre el formato UDO y el esquema Dublin Core (dc) o esquemas específicos de tesis (thesis).

\begin{center}
\begin{tabular}{|l|l|l|}
\hline
\textbf{Campo UDO} & \textbf{Campo DSpace} & \textbf{Notas} \\
\hline
Título / Subtítulo & dc.title / .alternative & Estándar \\
Autor(es) & dc.contributor.author & Incluir ORCID/Email \\
Palabras Claves & dc.subject & \\
Área / Línea Inv. & dc.subject.classification & O crear esquema local `udo` \\
Resumen & dc.description.abstract & \\
Tutor & dc.contributor.advisor & Rol `TU` en PDF \\
Jurado & dc.contributor.committeeMember & Rol `JU` (Requiere registro) \\
Fecha Aprobación & dc.date.issued & \\
Alcance Espacial & dc.coverage.spatial & \\
Alcance Temporal & dc.coverage.temporal & \\
Grado Asociado & thesis.degree.name & Requiere esquema `thesis` \\
Nivel Asociado & thesis.degree.level & Ej. Licenciatura, Maestría \\
Institución & dc.publisher & Universidad de Oriente \\
\hline
\end{tabular}
\end{center}

\subsection{2. Configuración del Formulario (input-forms.xml)}
Ejemplo de configuración para los campos de ``Alcance'' y ``Jurado'' definidos en la página 4 del formato UDO.

\begin{lstlisting}[language=xml, caption=Bloque XML para Alcance y Jurado]
<row>
    <!-- Campo: Alcance Espacial -->
    <field>
        <dc-schema>dc</dc-schema>
        <dc-element>coverage</dc-element>
        <dc-qualifier>spatial</dc-qualifier>
        <repeatable>true</repeatable>
        <label>Alcance Espacial</label>
        <input-type>onebox</input-type>
        <hint>Ej. Universal, Inespacial, o lugar geográfico</hint>
        <required>Requerido</required>
    </field>
    <!-- Campo: Jurado (Requiere haber registrado 'committeeMember' previamente) -->
    <field>
        <dc-schema>dc</dc-schema>
        <dc-element>contributor</dc-element>
        <dc-qualifier>committeeMember</dc-qualifier>
        <repeatable>true</repeatable>
        <label>Jurado</label>
        <input-type>name</input-type>
        <hint>Apellidos y Nombres del Jurado</hint>
        <required></required>
    </field>
</row>
\end{lstlisting}

\textbf{Nota sobre Despliegue:}
Al modificar archivos XML de configuración (\texttt{submission-forms.xml}, \texttt{input-forms.xml}), es necesario reiniciar el contenedor del backend para recargar los cambios:
\begin{lstlisting}[language=bash]
docker compose restart dspace-server
\end{lstlisting}

\section{Solución de Problemas Comunes}

\subsection{Cambios en input-forms.xml no visibles}
Si ha modificado el archivo de formularios pero la interfaz web sigue mostrando la versión anterior, verifique lo siguiente:

\begin{enumerate}
    \item \textbf{Reinicio del Backend:} DSpace carga la configuración XML en memoria al inicio. Es obligatorio reiniciar el servicio para aplicar cambios:
    \begin{lstlisting}[language=bash]
docker compose -p dspace restart dspace-server
    \end{lstlisting}
    \item \textbf{Nombre del Archivo:} Asegúrese de que el archivo se llame exactamente \texttt{input-forms.xml}. Errores tipográficos (ej. \texttt{unput-forms.xml}) harán que el sistema ignore el archivo.
    \item \textbf{Sintaxis XML:} Un error de sintaxis puede provocar que DSpace descarte el archivo y use la configuración por defecto. Verifique los logs:
    \begin{lstlisting}[language=bash]
docker compose -p dspace logs -f dspace-server
    \end{lstlisting}
\end{enumerate}

\section{Áreas de Asistencia}
\begin{itemize}
    \item \textbf{Instalación:} Despliegue con Docker o instalación manual (Maven/Ant).
    \item \textbf{Personalización:} Modificar el tema de Angular o crear nuevos componentes.
    \item \textbf{REST API:} Cómo consumir la API para integraciones externas.
    \item \textbf{Migración:} Pasar de DSpace 6 (XMLUI/JSPUI) a DSpace 7/8.
    \item \textbf{OAI-PMH:} Configuración para la recolección de metadatos.
\end{itemize}

\end{document}